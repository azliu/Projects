%
% latex-sample.tex
%
% This LaTeX source file provides a template for a typical research paper.
%

%
% Use the standard article template.
%
\documentclass{article}

% The geometry package allows for easy page formatting.
\usepackage{geometry}
\geometry{letterpaper}

% Load up special logo commands.
\usepackage{doc}

% make a reference to Hypertext 
\usepackage{hyperref}

% Package for formatting URLs.
\usepackage{url}

% Packages and definitions for graphics files.
\usepackage{graphicx}
\usepackage{epstopdf}
\DeclareGraphicsRule{.tif}{png}{.png}{`convert #1 'dirname #1'/'basename #1 .tif'.png}

%
% Set the title, author, and date.
%
\title{Caesar Staffing Problem  \\ \small{DNSC 6211: Programming for Analytics}}
\author{
    Akash Bose \\
	Yatao Lu \\
	Xinyi Lu \\
	Zhencun Liu \\
}
\date{}

%
% The document proper.
%
\begin{document}

% Add the title section.
\maketitle

% Add an abstract.
\abstract{
Our team is trying to forcast a staffing model for front desk at Caesars Hotel in Las Vegas, by exploring available data to predict the consumer traffic. We choose this project because it is an existing problem not only for this specific hotel but for any other hotels where the staff is unionized. This predicting model can be applied industry-wide. To execute this project, we aquired a set of real-life data from Caesars Hotel including 8-month daily check-ins and check-outs. We imported the data set into Python, along with utilizing R, we develop a multivariate linear regression model to forecast the amount of total check-ins per day, and translate the daily check-in forecast into the number of staff members needed at the front desk.  
}

% Add various lists on new pages.
\pagebreak
\tableofcontents


% Start the paper on a new page.
\pagebreak

%
% Body text.
%
\section{Introduction}
\label{introduction}

For almost any hotel, ensuring sufficient front desk staff is crucial to the customer experience, in addition, front desk staff at hotels in United States are usually Union members, they work part-time and need their work schedule settled two weeks in advance, according to Union requirements, thus how to staff the front desk to improve customer experience is an important management and financial problem. If the front desk is under-staffed, guests will experience longer check-in time, which will impact their experience and impression at the hotel; if the front desk is overstaffed, hotels will usually suffer from higher cost. Being an industry-wide issue, we find that solving staffing problem for hotel industry can be quite practical and useful. 

\section{Background}

Explain your "storyline" and the relevance of the different components. Talk about the nature of the data; why you chose the data you did; how did looking at the data help you decide which dataset(s) to use; and, why? What were the questions that you started with? How did those questions change as you explored the data? Did you give up on some datasets because they were not interesting? Which ones were those? How did you finally end up getting to the question(s) that you finally answered? This entire process is will help me understand how you went about framing the issue(s). Limit this to 250 words.

\section{Method}

The overall question that you answered
Any sub-questions
Any questions that you had but could not respond to satisfactorily (this will not result in a negative grade - but will give me an idea of how you went about your project. Limit this to 250 words.

\section{Organization}

Describe how you divided up the work in your group. Limit this to 100 words.


\subsection{Workflow}

Provide a diagram of the workflow for your project. The command to include a diagram is shown below. Make sure you \underline{remove the comment} and \underline{change the name of the graphic file} without extension. Also, instead of \textbf{Quick Build} choose \textbf{PDFLaTex} from the dropdown option. Then generate the pdf with \textbf{View PDF}.



\begin{figure}[hb]
  \centering
%    \includegraphics[scale=0.5]{SampleWorkFlow}
  \caption{The project workflow}

\end{figure}


There are three main steps we executed. To pre-analysis exploration, we choose time variables to histogram check-in distribution. We also analyze dependent variables and independent variables both together and separately to see how they correlate with each other - with boxplot, we see the distribution of all independent variables and dependent variables; with multivariate regression, we see relationship among continuous variables. Following this pre-analysis exploration, we utilize R to develop the multiple linear regression model. This leads us to the final forecast and visualization. In this part, we first dotplot to visualize coefficient strength. With the help of R-Shiny Interactive, we are able to explore continuous variables; and with the help of XLSX Forecast Models, we are able to develop time-series multivariate model.

\subsection{Project structure}

Describe your data sources. In addition, describe how they are related to each other and to the research question(s). Limit this to 250 words.

\subsection{Figures and Tables}

List your tables and figures and explain why you chose to use them. Explain how these tables and / or figures contribute to your "story." Limit this to 250 words.


\section{Discussion}

Our project focuses on a real industry case, which is how to help hotel better prepare their staffing model. We have over fifty independent variables so it is a difficult decision to figure out the key driver associated with the dependent variable, which is the number of daily check-ins. The biggest selling point is our utilization of multiple visualization tools, which help us to clearly see and identify the relationship between different variables and how strong they are correlated. For example, the histograms vividly show the number of check-ins associated with different year, month, day and even holidays. Thus from the graph we could know that Christmas has weigh more daily check-ins compared to the other holidays and we should place more from staff during this period. The second selling point is that we used both Time-series model and Casual relationship model to simulate and forecast the number of check-ins.

\subsection{Learnings}

We quite enjoyed the learning-by-doing process in order to show interactive and fun visualization to our peers. We have compared several visualization galleries, such as Bokeh and Lightening by Nate Silver. And finally we have chosen Bokeh as our main visualization tools because its gallery includes a lot of tools for statistical analysis. For example, after we finished all the regression, we used a dotplot method to show the coefficients of all the independent variables. The direction of the dot and the line implies if the variable is positive or negative related the the number of check-ins. The length of the line implies how far the variable will affect a unit of the dependent variable. Th negative line of Valentines Day could imply that the people are willing to spend the day at home. Therefore the hotel should note spend too much resource on this holiday.

\subsection{Challenges}

We have not learned how to do Time-series forecast so we got a lot of help from Professor Kanungo. The video and resources he gave us are very useful for us in completing our project. In the programming session, we have difficulty in applying new visualization packages to give different plots. Fortunately we discussed and helped each other and successfully make our codes working. Mostly importantly, solving the front desk staffing problem relied heavily on being able to accurately forecast guest arrivals on a daily basis, and then break that down by staffing shifts. However there were many variables, outlier events and uncertainties that made forecasting difficult. Therefore, we completed both Time-series and regression forecasts and compared both of their results.

\section{Bullets and numbered lists (FYI, delete in your report)}

This is up to you. If you want to add another section. This section explains how to make lists. In you final report you should delete this part. 

\subsection{Bulleted and Numbered Lists}

\LaTeX\ is very good at providing clean lists.  Examples are shown below.

\begin{itemize}
\item Bulleted items come out properly indented and spaced, every time.

\begin{itemize}
\item Sub-bullets are a virtual no-brainer: just nest another \verb!itemize! block.
\item Note how the bullet character automatically changes too.
\end{itemize}

\item Just keep on adding \verb!\item!s\ldots

\item \ldots until you're done.
\end{itemize}

Numbered lists are almost identical, except that you specify \verb!enumerate! instead of \verb!itemize!.  List items are specified in exactly the same way (thus making it easy to change list types).

\begin{enumerate}
\item A list item
\item Another list item
\item A list item with multiple nested lists

\begin{itemize}
\item Nested lists can be of mixed types.
\item That's a lot of power and flexibility for the price of learning a handful of directives.

\begin{enumerate}
\item Like nested bullet lists, nested numbered lists also ``intelligently'' change their numbering schemes.
\item Meanwhile, all \emph{you} have to write is \verb!\item!.  \LaTeX\ does the rest.
\end{enumerate}
\end{itemize}

\item Back to your regularly scheduled list item

\end{enumerate}

BTW, this is a great site to generate tables in Latex and learn how to do it in Latex -- \url{http://www.tablesgenerator.com/}


\section{Conclusion}

Ensuring sufficient front desk staff is crucial to hotel guests' customer experience, and how to staff the front desk to improve customer experience is an important management and financial issue, thus this project of forecasting the number of total front desk staff is practical. By exploring a 8-month daily check-in/check-outs real-life dataset of Caesars Hotel, with the utilization of Python, R and Visualization tools, we are able to produce a forecast with in-sample average of 0.25 MAPE and out-sample average of 0.34.

\end{document}

